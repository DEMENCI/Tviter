\section{Technology Stack}
\begin{itemize}
	\item \textbf{Frontend} 
	\begin{itemize}
		\item Vue - Progressive JS Framework;
		\item Axios - HTTP client;
		\item Semantic UI - CSS Framework;
		\item VeeValidate - Input Validation for Vue.js;
	\end{itemize}
	
	\item \textbf{Backend}
	\begin{itemize}
		\item AdonisJS - Node.js MVC Framework; 
		\item Node.js 10.15.0;
		\item NPM 6.4.1;
		\item MySQL;
	\end{itemize}	
\end{itemize}

\section{Project implementation}
\subsection{Backend}
	As mentioned above for the backend we used AdonisJS, a Node.js MVC framework. Here are some characteristics of it : 
	\begin{itemize}
		\item Solid MVC arhitecture;
		\item Active Record based ORM;
		\item Unit testing API;
	\end{itemize}

	First we need to instal the Adonis CLI for future usages with the command:
	\begin{verbatim}
		npm install -g @adonis/cli
	\end{verbatim}
	
	Then, to create the application we used the command:
	\begin{verbatim}
		adonis new tviter_api --api-only
	\end{verbatim}
	
	Here was used the \textbf{--api-only} flag because we don't need any views from adonis, only the blueprint for the api.
	
	To launch the app run commands:
	\begin{verbatim}
		cd tviter_api
		adonis serve --dev
	\end{verbatim}
	
	It will be accessible at \textbf{localhost:3333}.
	
	Next comes the database setup. For the DBMS we used MySQL. To interact with the app we need to install the coresponding package:
	\begin{verbatim}
		npm install mysql --save
	\end{verbatim} 
	
	The details for the DB connection are located in the environment variables from .env:
	\begin{verbatim}
		DB_CONNECTION=mysql
		DB_HOST=127.0.0.1
		DB_PORT=3306
		DB_USER=root
		DB_PASSWORD=$%#@&
		DB_DATABASE=tviter
	\end{verbatim}
	
	After the database is configured, to create all the necessary tables and all of it we need migrations.
	
	To create a migration we use the command:
	\begin{verbatim}
		adonis make:migration name
	\end{verbatim} 
	
	In the current app there are 6 migrations all identified by an id and the table or object it creates/modifies: $id\_name.js$
	
	To keep the form of the data an the relations between tables we use models. The models used in the API are: 
	\begin{itemize}
		\item Favorite;
		\item Reply;
		\item Tweet;
		\item User;
	\end{itemize}
	
	In those are specified relations with methods from adonis API.
	
	For the Controller we have 3 Http request/response controllers created with the command:
	\begin{verbatim}
		adonis make:controller name --type=http
	\end{verbatim}
	
	These are:
	\begin{itemize}
		\item FavoriteController;
		\item TweetController;
		\item UserController;
	\end{itemize}

	Also, the routes are defined in the routes.js file from start folder.
	
\subsection{Frontend}
	The frontend was implemented in a Vue.js app. Initially we need Vue cli ofcourse and it can be installed with the command:
	\begin{verbatim}
		npm install -g vue-cli
	\end{verbatim}
	
	Then to create the app we used the command:
	\begin{verbatim}
		vue init webpack tviter-frontend
	\end{verbatim}
	
	To launch the app use the commands:
	\begin{verbatim}
		cd tviter-frontend
		npm install
		npm run dev
	\end{verbatim}
	
	To connect the frontend to the API we need axios which is an HTTP client. Install it with:
	\begin{verbatim}
		npm install axios --save
	\end{verbatim}
	
	Then to configure it go in src/main.js file and update as follows:
	\begin{verbatim}
		import axios from 'axios'
		
		// add these before Vue is instantiated
		window.axios = axios // reference axios globally
		axios.defaults.baseURL = 'http://localhost:3333' // default URL for API
	\end{verbatim}
	
	As CSS framwork we used SemanticUI which was include in index.html by CDN:
	\begin{verbatim}
		<link rel="stylesheet" href="https://cdnjs.cloudflare.com/ajax/libs/semantic-ui/2.2.13/semantic.min.css" />
	\end{verbatim}
	
	After this comes the UI. It was implemented with components. Each component contains the markup and the script. The used components can be arranged as a hierarchy in the following way:
	\begin{itemize}
		\item \textbf{Components}
		\item \begin{itemize}
			      \item \textbf{Auth}
			      \item LoginForm
			      \item SignupForm				
	   		  \end{itemize}
   		\item \begin{itemize}
   		          \item \textbf{Tweet}
   		          \item Replies
   		          \item SingleTweet
   		          \item Tweet
   		          \item TweetRecations
   		          \item Tweets
   			  \end{itemize}
   		\item \begin{itemize}
	   			  \item \textbf{User}
  	   			  \item \begin{itemize}
  	   			  	        \item \textbf{Profile}
  	   			  	        \item FavoriteTweets
  	   			  	        \item UserCard
  	   			  	        \item UserFollowers
  	   			  	        \item UserProfile
  	   			  	        \item UserProfileHeader
  	   			  	        \item UserProfileSidebar
  	   			  	        \item UsersFollowing    
  	   			        \end{itemize}  
	   			  \item \begin{itemize}
							\item \textbf{Settings}
							\item UserPasswordSettings
							\item UserProfileSettings
							\item UserSettingsMenu	
	        	   	    \end{itemize}
	   			  \item UserSidebar
	   			  \item WhoToFollow
	   		\end{itemize}
   		\item Home
   		\item Notification  
	\end{itemize}

	The routes are kept in router/index.js file.
\clearpage